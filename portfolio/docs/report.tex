\documentclass[12pt]{article}
\usepackage{graphicx} % For including images
\usepackage{hyperref} % For hyperlinks
\usepackage{geometry} % For adjusting page margins
\usepackage{xcolor} % For adding colors
\geometry{a4paper, margin=1in}

% Define formal colors
\definecolor{titlecolor}{RGB}{0, 51, 102} % Dark blue for title
\definecolor{sectioncolor}{RGB}{51, 51, 153} % Dark blue for sections
\definecolor{linkcolor}{RGB}{0, 102, 204} % Blue for links
\definecolor{textcolor}{RGB}{51, 51, 51} % Dark gray for text

% Set default text color
\color{textcolor}

\title{\color{titlecolor}Assignment CH02-00: Personal Portfolio}
\author{\color{titlecolor}BLOUL MOHAMED RIDHA ABDESSAMED}
\date{\color{titlecolor}\today}

\begin{document}

\maketitle

% Add table of contents with hyperlinks
\renewcommand{\contentsname}{\color{sectioncolor}Table of Contents}
\tableofcontents
\newpage

% Add GitHub, Website, Group, and Email information
\section*{\color{sectioncolor}Project Information}
\begin{itemize}
    \item \textbf{GitHub Repository:} \href{https://github.com/Bloul-Mohamed/Assignment-CH02-00.git}{\color{linkcolor}GITHUB}
    \item \textbf{Website Link:} \href{https://yourwebsite.com}{\color{linkcolor}URL}
    \item \textbf{Group Name:} G3
    \item \textbf{Group Members:} 
        \begin{itemize}
            \item BLOUL MOHAMED RIDHA ABDESSAMED (\href{mailto:m.bloul.inf@lagh-univ.dz}{\color{linkcolor}email})
        \end{itemize}
\end{itemize}

\section{\color{sectioncolor}Introduction}
\label{sec:introduction}
This report documents the process of creating a personal portfolio using HTML5 and CSS3 as part of the Web Development course assignment. The objective was to design a webpage showcasing my skills, projects, and contact information. The portfolio demonstrates my understanding of HTML5 semantic elements and CSS3 techniques such as Flexbox, Grid, transitions, and animations.

The portfolio was designed to be responsive, ensuring it works seamlessly on various devices, including desktops, tablets, and mobile phones. The project also emphasized clean and well-organized code, with meaningful comments to ensure maintainability.

\section{\color{sectioncolor}Design and Layout}
\label{sec:design}
The portfolio is structured into multiple pages, each serving a specific purpose:
\begin{itemize}
    \item \textbf{Homepage}: The homepage provides a brief introduction about myself, including my background, interests, and career goals. It serves as the main entry point for visitors.
    \item \textbf{Skills Section}: This page highlights my technical skills, including programming languages (e.g., HTML, CSS, JavaScript), tools (e.g., Git, VS Code), and frameworks (e.g., Bootstrap).
    \item \textbf{Projects Section}: This page showcases my projects and work experience. Each project includes a brief description, technologies used, and a link to the GitHub repository or live demo.
    \item \textbf{Contact Section}: This page includes a contact form where visitors can send me messages. The form includes fields for name, email, subject, and message, with client-side validation to ensure proper input.
\end{itemize}

The layout was designed using **CSS Grid** and **Flexbox** to ensure a responsive and visually appealing structure. The design is minimalistic, with a focus on usability and aesthetics. The color scheme was chosen to create a professional and modern look, with consistent typography and spacing.

\section{\color{sectioncolor}Features}
\label{sec:features}
The portfolio includes the following features:
\begin{itemize}
    \item \textbf{Responsive Design}: The layout adapts to different screen sizes using **media queries**. This ensures that the portfolio looks great on desktops, tablets, and mobile devices.
    \item \textbf{CSS Animations}: Subtle animations were added to buttons and links to enhance user interaction. For example, buttons change color when hovered over, and links have a smooth underline effect.
    \item \textbf{Contact Form}: A simple form was implemented using HTML5 form elements. The form includes client-side validation using attributes like \texttt{required} and \texttt{type="email"}.
    \item \textbf{Semantic HTML}: Proper use of HTML5 semantic tags like \texttt{<header>}, \texttt{<main>}, \texttt{<section>}, and \texttt{<footer>} was prioritized for better accessibility and SEO.
    \item \textbf{Cross-Browser Compatibility}: The portfolio was tested on multiple browsers (e.g., Chrome, Firefox, Edge) to ensure consistent behavior and appearance.
\end{itemize}

These features were chosen to create a modern and interactive user experience while adhering to best practices in web development.

\section{\color{sectioncolor}Challenges and Solutions}
\label{sec:challenges}
During the development process, I faced the following challenges:
\begin{itemize}
    \item \textbf{Responsive Design}: Ensuring the layout looked good on all devices was initially difficult. I overcame this by using **CSS Grid** and **media queries** to adjust the layout for different screen sizes. For example, the navigation menu switches to a hamburger menu on smaller screens.
    \item \textbf{Browser Compatibility}: Some CSS properties behaved differently across browsers. I resolved this by testing the portfolio on multiple browsers and using **vendor prefixes** where necessary.
    \item \textbf{Form Validation}: Implementing client-side validation for the contact form required research. I used HTML5 form validation attributes like \texttt{required} and \texttt{type="email"} to ensure proper input. Additionally, I added custom error messages to guide users.
    \item \textbf{Performance Optimization}: To improve the portfolio's performance, I optimized images and used **minified CSS** to reduce load times.
\end{itemize}

\section{\color{sectioncolor}Conclusion}
\label{sec:conclusion}
The personal portfolio project was a valuable learning experience. It allowed me to apply the HTML5 and CSS3 concepts covered in the course and explore advanced techniques like animations and responsive design. The final product is a functional and visually appealing portfolio that effectively showcases my skills and projects.

This assignment has significantly improved my understanding of modern web development practices, including the importance of clean code, responsive design, and accessibility. I also gained experience in troubleshooting and problem-solving, which will be invaluable in my future projects.

Overall, the portfolio project has been a rewarding experience, and I am proud of the final result. I look forward to further refining my skills and taking on more challenging projects in the future.

\end{document}
